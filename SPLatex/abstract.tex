%   Filename    : abstract.tex 
\begin{abstract}
This research collates results from myriad studies which tackled mental health awareness, issues, and mitigations among students. Numerous studies have concluded that proper mental health consultations must be done to mitigate and address psychological issues experienced by most students during the pandemic and online learning setup. In the University of the Philippines Visayas, the counseling appointments are scheduled through multiple social media sites, and the communication modalities between the counselors and students are thru video/phone call, chat, or text. These online platforms are outside the range of the university's ownership as personal social media accounts are being used for meetings that might be vulnerable to cyber threats and/or risks. Communicating with the counselors enabled researchers to identify what should be the main features of the system website that would be implemented. In this paper, the researchers are dedicated to creating a website for guidance and counseling system that is inclusive for the University of the Philippines Visayas. The implementation of the website would be a step closer to the goal of awareness and mitigation of mental health problems in the community as well as strengthening the community bond.

\begin{flushleft}
\begin{tabular}{lp{4.25in}}
\hspace{-0.5em}\textbf{Keywords:}\hspace{0.25em} & Guidance and Counseling System, Mental health, Mental health awareness, website\\
\end{tabular}
\end{flushleft}
\end{abstract}
